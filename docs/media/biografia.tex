% Options for packages loaded elsewhere
\PassOptionsToPackage{unicode}{hyperref}
\PassOptionsToPackage{hyphens}{url}
%
\documentclass[
]{article}
\usepackage{lmodern}
\usepackage{amsmath}
\usepackage{ifxetex,ifluatex}
\ifnum 0\ifxetex 1\fi\ifluatex 1\fi=0 % if pdftex
  \usepackage[T1]{fontenc}
  \usepackage[utf8]{inputenc}
  \usepackage{textcomp} % provide euro and other symbols
  \usepackage{amssymb}
\else % if luatex or xetex
  \usepackage{unicode-math}
  \defaultfontfeatures{Scale=MatchLowercase}
  \defaultfontfeatures[\rmfamily]{Ligatures=TeX,Scale=1}
\fi
% Use upquote if available, for straight quotes in verbatim environments
\IfFileExists{upquote.sty}{\usepackage{upquote}}{}
\IfFileExists{microtype.sty}{% use microtype if available
  \usepackage[]{microtype}
  \UseMicrotypeSet[protrusion]{basicmath} % disable protrusion for tt fonts
}{}
\makeatletter
\@ifundefined{KOMAClassName}{% if non-KOMA class
  \IfFileExists{parskip.sty}{%
    \usepackage{parskip}
  }{% else
    \setlength{\parindent}{0pt}
    \setlength{\parskip}{6pt plus 2pt minus 1pt}}
}{% if KOMA class
  \KOMAoptions{parskip=half}}
\makeatother
\usepackage{xcolor}
\IfFileExists{xurl.sty}{\usepackage{xurl}}{} % add URL line breaks if available
\IfFileExists{bookmark.sty}{\usepackage{bookmark}}{\usepackage{hyperref}}
\hypersetup{
  pdftitle={Biografia},
  pdfauthor={Ben Dêivide},
  hidelinks,
  pdfcreator={LaTeX via pandoc}}
\urlstyle{same} % disable monospaced font for URLs
\usepackage[margin=1in]{geometry}
\usepackage{graphicx}
\makeatletter
\def\maxwidth{\ifdim\Gin@nat@width>\linewidth\linewidth\else\Gin@nat@width\fi}
\def\maxheight{\ifdim\Gin@nat@height>\textheight\textheight\else\Gin@nat@height\fi}
\makeatother
% Scale images if necessary, so that they will not overflow the page
% margins by default, and it is still possible to overwrite the defaults
% using explicit options in \includegraphics[width, height, ...]{}
\setkeys{Gin}{width=\maxwidth,height=\maxheight,keepaspectratio}
% Set default figure placement to htbp
\makeatletter
\def\fps@figure{htbp}
\makeatother
\setlength{\emergencystretch}{3em} % prevent overfull lines
\providecommand{\tightlist}{%
  \setlength{\itemsep}{0pt}\setlength{\parskip}{0pt}}
\setcounter{secnumdepth}{-\maxdimen} % remove section numbering
\ifluatex
  \usepackage{selnolig}  % disable illegal ligatures
\fi

\title{Biografia}
\usepackage{etoolbox}
\makeatletter
\providecommand{\subtitle}[1]{% add subtitle to \maketitle
  \apptocmd{\@title}{\par {\large #1 \par}}{}{}
}
\makeatother
\subtitle{Um pouco sobre mim}
\author{Ben Dêivide}
\date{21/03/2021}

\begin{document}
\maketitle

\hypertarget{biografia}{%
\subsection{Biografia}\label{biografia}}

Sou natural de \href{http://pt.wikipedia.org/wiki/Pau_dos_Ferros}{Pau
dos Ferros}, RN, que fica a uma distância de 450km da capital, Natal,
onde ocorreu minha trajetória estudantil até o ensino médio. Em 2001, no
primeiro ano do ensino médio, para conseguir minha bolsa de estudos,
trabalhei como monitor de inglês no ensino fundamental na escola
Educandário Imaculada Conceição. Descobri nesse período, o prazer em
ensinar. Apesar de pouco conhecimento da língua inglesa, a preocupação
em dar uma boa aula, me motivava a estudar e mostrava o quanto o ensino
era fascinante. Ao término do ensino médio, em 2003, tive outra
oportunidade de lecionar física e inglês, na Escola Municipal
Prof.~Severino Bezerra, em Pau dos Ferros/RN, no ensino fundamental II.

Em 2005, inicio o curso de graduação em Engenharia Agronômica, pela
Universidade Federal Rural do Semiárido (UFERSA), antiga ESAM, em
Mossoró, RN. Em virtude das condições financeiras não ser favorável,
tive que trabalhar em 2006, paralelamente ao curso de graduação em
Agronomia, para compor a renda do mês. Unindo o prazer com o trabalho,
consegui uma vaga de Professor para lecionar as disciplinas de Física e
Inglês na Escola Estadual José Martins de Vasconcelos, em Mossoró/RN, e
assim, arcar com custos financeiros. Durante o período acadêmico, me
identifiquei com a disciplina de Estatística Básica, e no terceiro
semestre do curso de Agronomia, concorri a tão sonhada bolsa de
monitoria em Estatística Básica. Daí em diante, vi que o caminho da
Estatística levava a um mundo extraordinário. Nesse pequeno intervalo,
tive também a experiência de lecionar em um curso básico de informática,
na empresa Ingetec Informática, ainda em Mossoró/RN. Percebi uma outra
paixão nos estudos, que era a área de programação.

Em 2010, concluí minha graduação. Nessa ocasião apresentei o trabalho de
conclusão do curso de Agronomia, que era intitulado ``Modelagem
estocástica da temperatura média da cidade de Mossoró, RN''. Sob a
orientação do
\href{https://sigaa.ufersa.edu.br/sigaa/public/docente/portal.jsf?siape=396304}{Prof.~Janilson
Pinheiro de Assis}, avaliamos seis modelos de distribuições (Normal,
Log-Normal, Beta, Gama, Log-Pearson, Gumbel e Weibull) de probabilidade
para verificar qual destes se ajustaria aos dados de temperatura média.
Para verificar a aderência das distribuições à série de temperatura
média foram utilizados testes de aderência (Qui-quadrado,
Kolmogorov-Smirnov, Cramér-von Mises, Anderson-Darling, Kuiper). A
distribuição que obteve melhor ajuste aos dados foi a distribuição
Normal em três escalas utilizadas.

Nesse mesmo ano, prestei concurso temporário ao
\href{https://www.ibge.gov.br/}{IBGE}, para participar do Censo 2010.
Aprovado como supervisor, trabalhei diretamente com 20 recenseadores, do
qual minha função era averiguar se a coleta de dados feita pelos
recenseadores estava correta. Uma experiência incrível, pois foi meu
primeiro trabalho fora da Universidade que estava ligado diretamente com
a estatística.

Determinado e focado no que queria, resolvi fazer pós-graduação em
Estatística tendo como segunda opção a área de Climatologia. Ainda em
2010, prestei concurso para quatro instituições de ensino: Universidade
Federal de Lavras - UFLA (Pós-Graduação em Estatística), Universidade
Federal de Viçosa - UFV (duas seleções: Pós-Graduação em Estatística
Aplicada e Biometria e outra seleção na Pós-Graduação em Meteorologia
Aplicada), Universidade Federal de Campina Grande - UFCG (Pós-Graduação
em Meteorologia), UFRN (Pós-Graduação em Matemática Aplicada e
Estatística). Fui aprovado em todas as seleções, sendo que na UFRN fiz
apenas a primeira etapa, devido ter obtido alguns resultados positivos
nas demais seleções e já está convicto para onde me destinar, resolvi
não prosseguir. Essa escolha, já havia sido tomada bem antes das provas
de seleções, quando comecei a ler as obras do Prof.
\href{https://des.ufla.br/~danielff/}{Daniel Furtado Ferreira}, e me vi
fascinado com os seus trabalhos, principalmente com o software
\href{https://des.ufla.br/~danielff/programas/sisvar.html}{SISVAR}. Daí,
decidi que o local onde iria fazer o mestrado seria a Universidade
Federal de Lavras, em Minas Gerais. Mesmo com muitos desafios, encarei
com muita naturalidade esse momento. Houve um fato interessante para
concorrer à seleção de Pós-Graduação da UFLA. Me deparei com a
dificuldade em saber como iria fazer a prova, pois nunca havia saído de
meu estado de origem. Assim, consegui por meio de uma solicitação, que a
prova fosse realizada na UFRN, em Natal, sendo que o Professor
responsável pela aplicação da prova foi Paulo César Formiga Ramos.

Fui aprovado para o mestrado em Estatística e Experimentação
Agropecuária e, ainda, sob a orientação do Prof.~Daniel Furtado. Em
2011, viajo para Lavras, com a minha esposa,
\href{https://www.instagram.com/allannadvl/}{Allanna Lopes}, para a
realização de um sonho. Essa fase foi um dos momentos mais difíceis em
minha vida, juntamente com a minha esposa. Nunca havíamos saído de nosso
estado, RN, e ainda mais dois jovens com 25 e 21 anos, respectivamente,
com sonhos diferentes, mas unidos pelo amor. Devo confessar, que ao
longo desses anos o melhor presente que recebi foi ter a oportunidade de
estar ao lado de Allanna. Em todos esses momentos, depois de estarmos
juntos, ela foi a minha fortaleza para suprir a ausência de ter estado
longe da minha família, para realizar um sonho. E ainda mais, o
sacrifício de Allanna nesse momento foi tão árduo, que teve de abdicar
os seus sonhos para está ao meu lado. Tenho a minha eterna gratidão, e o
fruto disso, é a nossa linda filha Maria Isabel, a pessoa que nutre o
nosso presente. Tenho certeza, que nos conhecemos de outros momentos, e
esta existência foi um reencontro. Sou muito feliz por isso.

Confesso que nesse período, sob a orientação do Prof.~Daniel, o
aprendizado sobre a estatística foi fundamental. E ele se torna nesse
momento, e até hoje, a minha referência como profissional na área da
Estatística. Em 2012, concluí o mestrado, intitulado
\href{http://repositorio.ufla.br/bitstream/1/753/1/DISSERTA\%c3\%87\%c3\%83O_Distribui\%c3\%a7\%c3\%a3o\%20exata\%20da\%20midrange\%20estudentizada\%20externamente\%20da\%20normal\%20e\%20desenvolvimento\%20de\%20uma\%20biblioteca\%20R\%20utilizando\%20Quadratur.pdf}{Distribuição
exata da midrange estudentizada externamente da normal e desenvolvimento
de uma biblioteca R utilizando Quadratura Gaussiana}. Muitos das ideias
nesse momento se concretizavam, que era entender um pouco mais sobre a
estatística e a programação, usando o ambiente
\href{http://r-project.org/}{R}, bem como técnicas de análise numérica,
simulação, dentre outros assuntos. Foi muito enriquecedor esse momento.

Além disso, consegui obter uma mudança de nível, para o Doutorado no
mesmo programa e com a continuação da orientação do Prof.~Daniel, sendo
finalizado em Maio de 2016, cuja tese foi
\href{http://repositorio.ufla.br/bitstream/1/11466/2/TESE_Testes\%20de\%20compara\%c3\%a7\%c3\%b5es\%20m\%c3\%baltiplas\%20baseados\%20na\%20distribui\%c3\%a7\%c3\%a3o\%20da\%20midrange\%20estudentizada\%20externamente.pdf}{Testes
de comparações múltiplas baseados na distribuição da midrange
estudentizada externamente}. Desde o mestrado, eu vinha estudando
juntamente com Daniel, desenvolvimento de pacotes. No doutorado, os
estudos começaram a se aprofundar, do qual, hoje é uma das minhas linhas
de pesquisa, desenvolver pacotes para a ciência de dados.

Na sequência, finalizo com o Pós-Doutorado em Estatística, ainda sob a
orientação do Prof.~Daniel, do qual foi finalizado em 2019, intitulado
\emph{Ferbat's test: A Monte Carlo multiple comparison procedure with a
control}. O teste Ferbat é uma homenagem aos nossos sobrenomes, Ferreira
e Batista. O artigo está submetido sob avaliação.

Em 2016, ingresso na Universidade Federal de São João del-Rei, em São
João del-Rei/MG, na carreira de docência no serviço público, como
Professor substituto, durante dois anos. Foi uma experiência incrível.
Em 2018, ingresso na mesma universidade, agora como professor efetivo no
campus Alto Paraopeba, Ouro Branco/MG, do qual estou até hoje. Confesso
que a fase de concursos, para mim, foi uma experiência que aprendi muito
a trabalhar a condição psicológica. E você que me ler, pode estar nessa
fase. O conselho é, acredite em você! Todos irão questionar o seu
propósito, mas só você realmente saberá a sua escolha. Insista, persista
e acredite. Saiba que em algum momento, você mesmo irá desacreditar. Não
se abale, é apenas um momento. Quando esse momento passar, retorne ao
seu foco novamente. Disciplina, disciplina e disciplina.

Toda essa história não seria possível se o casal,
\href{https://www.facebook.com/josealcigeriobatista.alcigerio}{Léo
Batista} e Chica (Pais), não tivesse o esforço para me proporcionar uma
estrutura mínima até esse momento. Juntamente, com meu único irmão,
\href{https://www.instagram.com/alefebatistareal/}{Álefe Batista},
formamos um quarteto mágico, nós nos amamos! Agraço imensamente, por ter
a oportunidade de compartilhar essa existência com vocês.

Aos demais membros da família, agradeço por meio do patriarca Alcides
Batista (Avó, \emph{in memoriam}). Por ter convivido com ele
intensamente, em toda a minha infância e juventude, não tem como ser um
espelho, desde a sua \emph{gambiarra} até a sabedoria.

Posso não ter externado tudo o que sinto por toda a minha trajetória
pessoal e profissional. Porém, com linhas tortas, tento levar a vida
sempre fazendo o meu melhor naquilo que acredito, sem objetivos, apenas
vivendo mais um dia como se fosse o último!

\end{document}
