\documentclass[landscape,a4paper]{article}
\usepackage{multicol, fancyhdr}
\usepackage[utf8]{inputenc}
\usepackage[T1]{fontenc}
\usepackage[landscape,margin=0.5in]{geometry}
\usepackage[mmddyyyy]{datetime}
\usepackage[table]{xcolor}

\begin{document}
	\begin{center}
		\bf \textsc{\LARGE Plano de Aula}\\ % Title
	\end{center}
	
	\begin{tabular}{l}
		Universidade Federal de São João del-Rei - UFSJ\\
		Concurso Público para Professor Efetivo - Edital XX/2022\\
		Professor: \emph{Ben Dêivide}\\
		Inscrição: XXXXXXXXXX\\
		Tema: Probabilidade Condicional, Independência e Teorema de Bayes\\
		Duração: 55 minutos\\
		Data: \rule{.5cm}{.1mm}/\rule{.5cm}{.1mm}/\rule{.5cm}{.1mm}
	\end{tabular}
	
	\begin{center}
		\begin{tabular}{|p{6cm}|p{6cm}|p{6cm}|p{6cm}|}
			\hline
			\large{Conteúdo} \cellcolor{lightgray} & \large{Situação didática} \cellcolor{lightgray} & \large{Metodologia} \cellcolor{lightgray} &
			\large{Avaliação} \cellcolor{lightgray} \\
			\hline
		\end{tabular}\\
		\begin{tabular}{|p{6cm}|p{6cm}|p{6cm}|p{6cm}|}
			\hline
			\begin{enumerate}
				\item Probabilidade Condicional
				\item Independência de Eventos
				\item Teorema da Probabilidade total
				\item Teorema de Bayes
			\end{enumerate} &
			\begin{itemize}
				\item Revisão sobre espaço amostral;
				\item Motivação;
				\item Definição da probabilidade condicional
				\item Mostrar que a probabilidade condicional também é uma medida de probabilidade;
				\item Independência de eventos;
				\item Resultados importantes para o Teorema de Bayes:
				\begin{itemize}
					\item Partição do espaço amostral;
					\item Teorema da probabilidade total;
				\end{itemize}
				\item Teorema de Bayes.
			\end{itemize}&
			Aula áudio/visual interativa com o auxílio dos seguintes recursos:
			\begin{itemize}
				\item Quadro/pincel;
				\item Projetor multimídia (Quando disponível);
				\item Material auxiliar.
			\end{itemize} &
			\begin{itemize}
				\item Questionamentos no decorrer da aula;
				\item Atividade escrita;
				\item Atividade de pesquisa (extra-classe).
			\end{itemize}\\
			\hline
		\end{tabular} \\[.4in]
	\end{center}
	\noindent {\Large Referências Bibliográficas}\\
	\noindent MAGALHÃES, M. N. {\bf Probabilidade e Variáveis aleatórias}.2. ed. São Paulo: EdUSP, 2006. 428p.\\
	MOOD, A. M.; GRAYBILL, F. A.; BOES, D. C. {\bf Introduction to the theory of Statistics}. New York: McGraw-Hill, 1974. 564p.\\
	CASELLA, George; BERGER, Roger L. { \bf Statistical Inference}. 2. ed. New York: Duxbury Press, 2001
	\vspace{1cm}
	\begin{center}
		\rule{10cm}{.1mm}\\
		Ben Dêivide de Oliveira Batista - Professor
	\end{center}
\end{document} 